% !TeX spellcheck = pl_PL
\chapter{Projekt}\label{ch:project}

\section{Prototyp interfejsu}\label{sec:mockups}

\lipsum[5]

\section{Opis podstawowej architektury systemu}\label{sec:basicArchitecture}

\lipsum[5]

\section{Projekt bazy danych}\label{sec:database}

\subsection{Kategorie}\label{subsec:database:categories}

\begin{enumerate}[label={\textbf{KAT/\protect\threedigits{\theenumi}}}, wide, labelwidth=!, labelindent=0pt, labelsep=0pt, series=reqs]
    \setlength\itemsep{1.75em}
    \req{User}\label{kat:User} (Użytkownik)\\
    \indent\textbf{Opis}: Konto użytkownika aplikacji. Każdy zalogowany użytkownik musi mieć konto użytkownika.
    \par
    \textbf{Atrybuty}:
    \begin{itemize}[series=atr, wide, align=left, leftmargin=190pt]
        \atr{id}\label{kat:User:id}- identyfikator
        \atr{login}\label{kat:User:login}- login użytkownika
        \atr{passwordHash}\label{kat:User:passwordHash}- reprezentacja hasła utworzona przez nałożenie na hasło funkcji skrótu
    \end{itemize}

    \req{Authority}\label{kat:Authority} (Rola)\\
    \indent\textbf{Opis}: Rola użytkownika od której zależy zakres uprawnień użytkownika.
    \par
    \textbf{Atrybuty}:
    \begin{itemize}[series=atr, wide, align=left, leftmargin=190pt]
        \atr{name}\label{kat:Authority:name}- nazwa roli
    \end{itemize}
\end{enumerate}

\subsection{Reguły funkcjonowania}\label{subsec:database:functionalRules}

\begin{itemize}[label={\textbf{Reguły dla}}, wide, labelwidth=!, labelindent=0pt]
    \setlength\itemsep{1.75em}
    \item\ref{kat:User}\mynobreakpar
    \begin{enumerate}[label={\textbf{REG/\protect\threedigits{\arabic{enumi}}}}, wide, labelwidth=!, align=left, leftmargin=3cm]
        %Relacje
        \item Użytkownik (\ref{kat:User}) musi mieć przynajmniej jedną rolę (\ref{kat:Authority}).
        \item Użytkownik (\ref{kat:User}) może mieć wiele ról (\ref{kat:Authority}).
        %CRUD
        \item \role{Gość} może dodawać nowego użytkownika (\ref{kat:User}).
        \item \role{Użytkownik} może wyświetlać, edytować i~usuwać swoje dane użytkownika (\ref{kat:User}).
        \item \role{Administrator} może wyświetlać i~usuwać dane użytkownika (\ref{kat:User}).
    \end{enumerate}
    \item\ref{kat:Authority}\mynobreakpar
    \begin{enumerate}[label={\textbf{REG/\protect\threedigits{\arabic{enumi}}}}, wide, labelwidth=!, align=left, leftmargin=3cm, resume]
        %Relacje
        %CRUD
        \item \role{Administrator} może dodawać, wyświetlać, edytować i~usuwać dane roli (\ref{kat:Authority}).
    \end{enumerate}
\end{itemize}

\subsection{Ograniczenia dziedzinowe}\label{subsec:database:restrictions}

\begin{itemize}[label={\textbf{Ograniczenia dla}}, wide, labelwidth=!, labelindent=0pt]
    \setlength\itemsep{1.75em}
    \item\ref{kat:User}\mynobreakpar
    \begin{enumerate}[label={\textbf{OGR/\protect\threedigits{\arabic{enumi}}}}, wide, labelwidth=!, align=left, leftmargin=3cm]
        %Required
        \item Atrybut \ref{kat:User:login} jest wymagany.
        \item Atrybut \ref{kat:User:passwordHash} jest wymagany.
        %Unique
        \item Atrybut \ref{kat:User:login} ma unikalną wartość.
        %Type
        \item Atrybut \ref{kat:User:login} jest ciągiem znaków składającym się z~liter, cyfr i~dodatkowo mogącym zawierać znaki ".", "\_", "-", "@" o~długości od 1~do 50 znaków.
        \item Atrybut \ref{kat:User:passwordHash} jest ciągiem znaków o~długości 60 znaków.
    \end{enumerate}
    \item\ref{kat:Authority}\mynobreakpar
    \begin{enumerate}[label={\textbf{OGR/\protect\threedigits{\arabic{enumi}}}}, wide, labelwidth=!, align=left, leftmargin=3cm, resume]
        \item Atrybut \ref{kat:Authority:name} jest wymagany.
        \item Atrybut \ref{kat:Authority:name} ma unikalną wartość.
        \item Atrybut \ref{kat:Authority:name} jest ciągiem znaków składającym się z~liter i~znaków "\_" o~długości od 1~do 255 znaków.
    \end{enumerate}

\end{itemize}

\subsection{Model informacyjny}\label{subsec:database:domainModel}

\lipsum[5]

\thispagestyle{normal}
